\chapter{Conclusions and future work}
\label{ch:conclusions}
Evaluating \ac{PX} is important to assess the impact on patients' motivation when \acp{PREG} are used as part of physical rehabilitation treatments. Regarding, \acp{PREG}, physiotherapists' expectations are related to offering a compelling and safe activity for patients, while patients' major expectation is related to the therapeutic value of \acp{PREG} since they are interested in finishing rehabilitation treatments as soon as possible. Thus, \acp{PREG} should be evaluated considering both entertaining and therapeutic goals. In this work, a \ac{PX} model to evaluate \acp{PREG} comprehensively is presented. The model is composed of three layers of abstraction; i.e., the context, the player/patients and the game system. Also, it suggests that \ac{PX} should be evaluated regarding three moments; i.e., the antecedents, the interaction and the effects. The antecedents are everything that shapes the motivation to start or continue playing a \ac{PREG} (e.g. patients' and physiotherapists' expectations), the interaction is the moment when patients play a \ac{PREG}, and the effects are the consequences of the interaction that affect future gameplay episodes (e.g. short and long-term motivation).

The model was built considering current research in \ac{PX} and the physiotherapists' point of view. In the study presented in \autoref{ch:characterising}, we concluded that \acp{PREG} should be evaluated considering constraints and characteristics associated with the rehabilitation context, the therapeutic goal and the patients' characteristics. Similarly, the physiotherapists allowed us to identify aspects that should be evaluated to meet those constraints (e.g. rehabilitation goal support, configuration capability, tutorial quality and aspects to assess patients' physical progress). Those aspects and corresponding instruments and methods are presented in \autoref{ch:aspects}. That confirmed the importance of taking the three layers of abstraction into account. Additionally, the identified constraints confirmed that motivation to play/use a \ac{PREG} is not only shaped by the entertaining purpose but also by the way it assists patients' recovery process and physiotherapists work. Consequently, antecedents and effects are highly important when evaluating \acp{PREG}.

The purpose of the model is to guide process of evaluating \acp{PREG}. To aim that, evaluation aspects, methods and instruments from the \ac{GUR}, \ac{UX} and physiotherapy domains were identified and mapped into the corresponding layer of abstraction and moment of the model (See \autoref{sec:methodology}); thus, facilitating evaluators' work when performing an evaluation. Nevertheless, each evaluation may require aspects, instruments and methods not identified in this research.

The model considers that \ac{PX} may be affected by previous patients' and physiotherapist's experiences, leading to consider \ac{PX} as feedback loop in which the effects of an interaction affect the antecedents of future gameplay episodes. As a result, evaluating one episode of iteration would be not enough to perform a comprehensive evaluation. Therefore, the model includes a methodology to perform evaluate \acp{PREG} iteratively (See \autoref{sec:methodology}).

Questionnaires are one of the most employed methods to evaluate \ac{PX}; however, the process to develop a questionnaire is not standardised yet. In \autoref{ch:questionnaire}, a standard to develop questionnaires is proposed. The standard is built considering the guidelines of ISO, the most relevant standardisation organisation in the field. The content of the standard integrates existing guidelines on questionnaire development and current practices in the \ac{UX} and \ac{PX}. We concluded that developing a questionnaire is a respondent centred and iterative process.

Physiotherapists' perception and participation are relevant in the use of \acp{PREG}. Their perception regarding the use of a \ac{PREG} may represent an antecedent facilitates or prevents its continuous use in a rehabilitation treatment. Although methods such as interviews may allow assessing physiotherapists' perception, those take time to conduct, which is undesirable since physiotherapists' workload limits their available time. In that context, a closed response questionnaire would allow performing an adequate evaluation and addressing the time limitation. Therefore, we developed the first version of a questionnaire to asses physiotherapist's perception regarding the use of a \ac{PREG} following the proposed standard (See \autoref{ap:physiotherapists_questionnaire}). The questionnaire should be further pre-tested to asses reliability and validity.

Finally, evaluating Playtherapy suggested that the proposed model and methodology may allow assessing \ac{PX} in \acp{PREG} comprehensively (See \autoref{ch:playtherapy}). Also, it demonstrated that this process is demanding since evaluating the three layers of abstraction throughout the three moments comprises a large set of aspects. Also, the evaluation process should be integrated along the whole development life cycle. Playtherapy's evaluation required several iterations involving several sessions with developers, physical rehabilitation experts and patients. The evaluation of antecedents was adequate since it allowed assessing Playtherapy's potential to be used with patients safely. Nonetheless, we were only able to asses short-term motivation. The physiotherapist and patients who participated in the evaluation expressed their willingness to use Playtherapy as part of their rehabilitation treatments. To evaluate the long-term effects of  Playtherapy, one should evaluate the interactions during the whole treatment of different patients.

In summary, this research allowed to propose a \ac{PX} model to evaluate \acp{PREG} from a theoretical and pragmatical perspective. Theoretically, we identified the elements that might shape \ac{PX} in the case of \acp{PREG}; and pragmatically, we proposed a methodology and identified a set of aspects, instruments a and methods that may be used to perform an evaluation. The model integrates current research in \ac{PX} with the constraints of \acp{PREG} representing a novel proposal. It may contribute to standardising the process of evaluation. Also, the model has a therapeutic value since it may represent a tool for developers to ensure the quality of \ac{PREG} required to be used as a safe physical rehabilitation tool.

\section{Limitations and future work}
The work reported in this document is still in the early stages. The model needs to be validated using more \acp{PREG} and with a greater number of participants. Since we employed a qualitative approach to identify relevant constraints and aspects and validate the model with a case study, our findings cannot be generalised. Although we tried to overcome this limitation considering different works from the literature, We still need to validate whether the identified aspects are sufficient to evaluate other \acp{PREG}. Additionally, the constraints that we employ as a base to built the model are mainly related to \acp{PREG} to be used under the supervision of a physiotherapist in a hospital; therefore, evaluating autonomous \acp{PREG} may imply different challenges that need to be addressed.

The model suggests that \ac{PX} evaluation should be conducted iteratively. The study case allowed us to follow this recommendation since the evaluated \ac{PREG} is developed in an agile development environment. However, the use of the model in non-agile environments is to be assessed.