% Abstract

\pdfbookmark[1]{Abstract}{Abstract} % Bookmark name visible in a PDF viewer

\begingroup
\let\clearpage\relax
\let\cleardoublepage\relax
\let\cleardoublepage\relax

\chapter*{Abstract} % Abstract name

Rehabilitation exergames are serious games that support physical rehabilitation treatments while providing a fun and engaging  experience to players. \ac{PX} denotes the personal experience of playing digital games, thus \ac{PX} involves subjective players perceptions. Furthermore, rehabilitation exergames should be designed and evaluated considering the impact of using non traditional input devices and limitations imposed by a player's pathology and rehabilitation environment. Therefore, \ac{PX} evaluation is a crucial process in order to develop compelling and effective rehabilitation exergames. Consequently, \ac{PX} evaluation process should be integrated into the game development life cycle. However, a standardised approach to perform such evaluation for rehabilitation exergames is not established. 

This document presents the development of a model for evaluating \ac{PX} in rehabilitation exergames. This model may contribute to standardisation of comprehensive \ac{PX} evaluation for rehabilitation exergames as a main process of the game development cycle. Furthermore, a software application to guide on the use of the model was developed.
\endgroup			

\vfill