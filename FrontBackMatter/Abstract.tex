% Abstract

\pdfbookmark[1]{Abstract}{Abstract} % Bookmark name visible in a PDF viewer

\begingroup
\let\clearpage\relax
\let\cleardoublepage\relax
\let\cleardoublepage\relax

\chapter*{Abstract} % Abstract name

Physical rehabilitation treatments require patients to perform slow and repetitive exercises to recover a lost function. Consequently, patients usually lack motivation towards their rehabilitation treatments, resulting in a longer or even incomplete recovery process. \acp{PREG} are serious games that support physical rehabilitation treatments while providing a fun and engaging experience to players. Thus, \acp{PREG} may improve and increase patients' motivation and engagement in physical rehabilitation. \ac{PX} denotes the personal experience of playing digital games, thereby involving subjective players perceptions. Furthermore, to provide a positive \ac{PX}, \acp{PREG} should be designed and evaluated considering the impact of using non-traditional input devices and the limitations imposed by a player's impairment and the rehabilitation environment. Therefore, \ac{PX} evaluation is a crucial process to deploy compelling and effective \acp{PREG}; and the \ac{PX} evaluation process should be integrated into the game development life cycle. However, a standardised model to perform this evaluation process is not established yet.

This document presents a model for evaluating \ac{PX} in \acp{PREG}. The model was developed along three stages. First, two qualitative studies were conducted using semi-structured interviews to identify \acp{PREG}' constraints and relevant evaluation aspects, methods and instruments to employ. Second, a preliminary model was designed based on eight existing \ac{PX} models. Finally, the findings of the first stage and the preliminary model were integrated to propose the final model.

The model was validated evaluating Playtherapy, a \ac{PREG} developed by the Multimedia and Computer Vision research group from Universidad del Valle and Evaristo Garc\'ia University Hospital in Cali Colombia. The evaluation showed that Playtherapy might be a promising \ac{PREG} since patients and physiotherapists expressed their motivation to continue playing/using it. Moreover, the validation process demonstrated that the model can be used to evaluate \ac{PX} comprehensively in a \ac{PX} along the whole development life cycle. Thus, the model may contribute to standardising the \ac{PX} evaluation process for \acp{PREG}. 
\endgroup			

\vfill